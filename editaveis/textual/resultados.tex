\chapter[Resultados Obtidos]{Resultados Obtidos}

\section{Defeitos Encontrados}

*TODO lista dos defeitos levantados por cada um

\section{Criticidade dos defeitos encontrados}

O nível de criticidade dos defeitos apontados pela equipe de verificação e validação pode ser
considerado baixo. Já que o modelo de dados analisado encontrava-se em um nível bem elevado de maturidade.
Desta forma as correções que foram apontadas foram apenas há nível de refatoramento de Entidades e atributos.
Descrição suscinta de relacionamentos entre enticedades.

\section{Necessidade de reparos e Ações corretivas}

A necessidade de execução dos reparos levantados pela equipe de verificação e validação se faz necessáro para evitar
a ocorrência de erros futuros no projeto devido ao não entendimento ou entendimento incompleto do modelo de dados.
Dado que quanto mais tempo um erro leva para ser descoberto em um projeto de software, maior será o custo para reparar o mesmo.
Faz se extremamente necessário o reparo dos erros neste momento.

\section{Responsáveis e prazos de correção}

Baseado no processo definido, o responsável pelas correções levantadas pela equipe de verificação e validação será a integrante Karine
Valença. No período de 48 horas com base no documento de revisão gerado pelos demais integrantes.

\section{Faça uma análise crítica dos resultados}

*TODO
